\documentclass[fontsize=11pt]{article}
\usepackage{amsmath}
\usepackage[utf8]{inputenc}
\usepackage[margin=0.75in]{geometry}

\title{CSC111 Assignment 1: Linked Lists}
\author{Zaina Azhar, Katherine Luo}
\date{\today}

\begin{document}
\maketitle

\section*{Part 1: Faster Searching in Linked Lists}

\begin{enumerate}

\item[1.]
Complete this part in the provided \texttt{a1\_part1.py} starter file.
Do \textbf{not} include your solution in this file.

\item[2.]
Complete this part in the provided \texttt{a1\_part1\_test.py} starter file.
Do \textbf{not} include your solution in this file.

\item[3.]
\begin{enumerate}

    \item[(a)]
    \textbf{Running Time Analysis:} \texttt{LinkedList} $\rightarrow\Theta(n \cdot m)$\\
    \begin{itemize}
        \item Let n $\in \mathbb{N}$. Let \texttt{linky = LinkedList([0, 1, ..., n - 1])} with a length n. Let \texttt{item = n - 1}\\
        
        \textbf{Running time of the \texttt{linky.\_\_contains\_\_} method:} 
        \item The while loop takes n iterations, and each iteration takes 1 step. Total = n steps
        \item \texttt{curr = self.\_first} takes 1 step
        \item $(n-1) \in$ \texttt{linky}, therefore the \texttt{return False} line never runs
        \item The total running time of the \texttt{linky.\_\_contains\_\_} method is $n + 1$ steps, which is $\Theta(n)$\\
        
        \textbf{Running time of the for loop:}
        \item The for loop takes m iterations. 
        \item Each iteration calls \texttt{linky.\_\_contains\_\_(n-1)}, which has a running time of $\Theta(n)$
        \item Therefore, the for loop takes $n \cdot m$ steps.
        \item The total running time would be $n \cdot m$ steps, which is $\Theta(n \cdot m)$\\
    \end{itemize}

    \item[(b)]
    \begin{enumerate}
        \item \textbf{Running Time Analysis: \texttt{MoveToFrontLinkedList} $\rightarrow\Theta(n + m)$}
        \begin{itemize}
            \item Let $n \in \mathbb{N}$. Let \texttt{linky = MoveToFrontLinkedList([0, 1, ..., n - 1])} with a length of n
            \item Let \texttt{item = n - 1}
            \item First analyze the running time of the \texttt{linky.\_\_contains\_\_(n-1)} call. \\
            
            \textbf{Analysis for the First Search Operation}
            \item n - 1 is the last node, so the else branch in the \texttt{\_\_contains\_\_} method is executed. 
            \item The while loop takes n iterations to reach n - 1. Each iteration takes constant time, and n - 1 is reassigned to \texttt{self.\_first}. This is a total of n steps.
            \item (n - 1) $\in $ \texttt{linky}, so the \texttt{return False} line never runs.
            \item The if and elif branches each take 1 step to evaluate, for a total of 2 steps.
            \item Therefore for the first search operation, it takes n + 2 steps, which is $\Theta(n)$.\\
            
            \textbf{Analysis for the Subsequent Search Operations}
            \item n - 1 has been moved to \texttt{self.\_first} after the first search operation
            \item The if condition is evaluated and takes 1 step.
            \item The elif condition is first evaluated and then returns True, which takes 2 steps.
            \item The else branch never executes since n - 1 is now \texttt{self.\_first}.
            \item Therefore for subsequent search operations, it takes 3 steps, which is $\Theta(1)$.
            \item This is the running time for each iteration of the for loop, excluding the first one.\\
            
            \textbf{Running Time of the For Loop}
            \item For iteration 1, the operation in the for loop has a running time of $\Theta(n)$, or n steps.
            \item For subsequent iterations (m - 1 iterations), the operation in the for loop has a constant running time, which repeats (m - 1) times for a total of (m - 1) steps
            \item The total running time is $n + (m - 1)$, which is $\Theta(n + m)$\\
        \end{itemize}
        
        \item \textbf{Running Time Analysis: \texttt{SwapLinkedList}}
        \begin{itemize}
            \item Let $n \in \mathbb{N}$. Let \texttt{linky = MoveToFrontLinkedList([0, 1, ..., n - 1])} with a length of n
            \item Let \texttt{item = n - 1}
            \item The running time of Heuristic 2 can be split into two cases: when $m \leq n$, and when $m > n$\\
            
            \textbf{Case 1}: \emph{Running time analysis when $m \leq n$}
            \item n - 1 is the last node, so the else branch in the \texttt{\_\_contains\_\_} method is executed.
            \item The assignment statement takes 1 step.
            \item The while loop iterates less and less each time \texttt{\_\_contains\_\_} is called, as n - 1 moves up in the list with each iteration. Thus, the while loop iterates (n - i) times, where i indexes through the list starting at 1 (for n - 1). Each iteration takes constant time, for a total of (n - i) steps.
            \item The for loop then iterates m times; i increases until it reaches m, as we know $m \leq n$. 
            \item Steps taken by the for loop:
            \begin{align}
                & \sum^{m}_{i=1}(n - i)\\
                &= n \cdot m + \frac{-m^{2} - m}{2}\\
                &= \frac{1}{2} \cdot 2mn - m^{2} - m
            \end{align}
            \item Thus, when $m \leq n$, the running time of this loop is $(\frac{1}{2} \cdot 2mn - m^{2} - m) + 1$, which is $\Theta(nm - m^2)$ \\
            
            \textbf{Case 2:} \emph{Running time analysis when $m > n$}
            \item Similar to Case 1, for each iteration of the for loop, the while loop in the \texttt{\_\_contains\_\_} method takes (n - i) steps, where i indexes through the list starting at 1 (for n - 1). 
            \item Since $n < m$, these steps will iterate n times until the last node is shifted to the beginning of the list.
            \item Steps taken by the for loop:
            \begin{align}
                & \sum^{n}_{i=1}(n - i)\\
                &= \frac{n^{2} - n}{2}
            \end{align}
            
            \item However, since $n < m$, any iterations after m has reached the length of the list n will be constant time. This is because (n - 1) has moved to the first node.
            \item So for any (m - n) iterations, it would take constant time as the elif branch would execute and return. Thus, this is $(m - n)$ steps.
            
            \item The total running time is $\frac{n^{2} - n}{2} + m - n$ steps, which is $\Theta(n^2 + m)$\\
        \end{itemize}
        
        \item \textbf{Running Time Analysis: \texttt{CountLinkedList} $\rightarrow\Theta(n + m)$}
        \begin{itemize}
            \item Let $n \in \mathbb{N}$. Let \texttt{linky = CountLinkedList([0, 1, ..., n - 1])} with a length of n
            \item Let \texttt{item = n - 1}
            \item First analyze the running time of the \texttt{linky.\_\_contains\_\_(n-1)} call. \\
            
            \textbf{Analysis for the First Search Operation}
            \item n - 1 is the last node, so the else branch in the \texttt{\_\_contains\_\_} method is executed. 
            \item The while loop takes n iterations to reach n - 1. Each iteration takes constant time. The access count of n - 1 increases and thus is reassigned to \texttt{self.\_first}. This is a total of n steps.
            \item (n - 1) $\in $ \texttt{linky}, so the \texttt{return False} line never runs.
            \item The if and elif branches each take 1 step to evaluate, for a total of 2 steps.
            \item Therefore for the first search operation, it takes n + 2 steps, which is $\Theta(n)$.\\
            
            \textbf{Analysis for the Subsequent Search Operations}
            \item n - 1 has been moved to \texttt{self.\_first} after the first search operation
            \item The if condition is evaluated and takes 1 step.
            \item The elif condition is first evaluated, increases the access count, and then returns True. This takes 3 steps.
            \item The else branch never executes since n - 1 is now \texttt{self.\_first}.
            \item Therefore for subsequent search operations, it takes 4 steps, which is $\Theta(1)$.
            \item This is the running time for each iteration of the for loop, excluding the first one.\\
            
            \textbf{Running Time of the For Loop}
            \item For iteration 1, the operation in the for loop has a running time of $\Theta(n)$, or n steps.
            \item For subsequent iterations (m - 1 iterations), the operation in the for loop has a constant running time, which repeats (m - 1) times for a total of (m - 1) steps
            \item The total running time is $n + (m - 1)$, which is $\Theta(n + m)$
        \end{itemize}
    \end{enumerate}
\end{enumerate}

\item[4.]
\begin{enumerate}
    \item [a)] Let lst be a list of numbers from 0, 1, 2, .., n - 1. Consider the following m sequences of search operations on lst, where $m > n$:
    \begin{verbatim}
        n = lst.__len__
        for _ in range(1, m + 1):
            if m <= n:
                lst.__contains__(n - m)
            else:
                lst.__contains__(0)
    \end{verbatim}
    \item[b)] \textbf{Running Time Analysis for Heuristic 1: \texttt{MoveToFrontLinkedList}}
    \begin{itemize}
        \item Let \texttt{lst = MoveToFrontLinkedList([0, 1, 2, ..., n - 1])} be a list of length n\\
        
        \textbf{If Branch}
        \item For each iteration of the for loop, the if condition will execute for the until $m = n$.
        \item For the first call to \texttt{lst.\_\_contains\_\_(n - m)}, the last node (n - 1) is called and moved to the front. The (n - 2) node is now the last node. 
        \item Each subsequent call to \texttt{lst.\_\_contains\_\_(n - m)} will once again call the last node, shift it to the beginning of the list, and make the (n - 2) node go to (n - 1).
        \item Therefore, the each time the method is called, n nodes are traversed.
        \item This iterates until $m = n$, or until the entire list has been looped through. This results in n iterations, which traverse n nodes each time, for a total of $n^2$ steps\\
        
        \textbf{Else Branch}
        \item Once $m > n$, the else branch will execute. This occurs after n iterations, thus the else branch is iterated through $(m - n)$ times.
        \item For each iteration, \texttt{lst.\_\_contains\_\_(0)} is called. After the first if branch iterations, the entire list is shifted over until it is ordered in the same way it started.
        \item Calling \texttt{lst.\_\_contains\_\_(0)} will call the first node in the list. This executes the elif branch in the \texttt{\_\_contains\_\_} method, which takes constant time. This gves a total of $(m - n)$ steps.\\
        
        \textbf{Total Running Time}
        \item Let $T_1$ represent the total running time of Heuristic 1
        \item Ignoring constant factors, the total running time is $T_1 = n^2 + (m - n)$.\\
    \end{itemize}
    \item[b)] \textbf{Running Time Analysis of Heuristic 2: \texttt{SwapLinkedList}}
    \begin{itemize}
    \item Let \texttt{lst = SwapLinkedList([0, 1, 2, ..., n - 1])} be a list of length n. Assume n is even.\\
        
        \textbf{If Branch}
        \item For each iteration of the for loop, the if condition will execute for the until $m = n$.
        \item For the first call to \texttt{lst.\_\_contains\_\_(n - m)}, the last node (n - 1) switches with the second last node (n - 2).
        \item The next call to \texttt{lst.\_\_contains\_\_(n - m)} will now call the second last node and swap it back to its original spot at the end of the list.
        \item Following this pattern, the actual number of nodes traversed decreases with every other iteration. After two consecutive nodes swap with each other, the \texttt{lst.\_\_contains\_\_(n - m)} call moves further up the list.
        \item Thus, the number of nodes traversed over all iterations of the if branch can be represented as:
        \begin{align}
            & \sum_{i=0}^{\frac{n}{2}}2(n - 2i)\\
            &= 2 \cdot \sum_{i=0}^{\frac{n}{2}} (n - 2i)\\
            &= 2n - \frac{n(n+2)}{2} + n^2
        \end{align}
        \item Thus, the if branch takes $2n - \frac{n(n+2)}{2} + n^2$ steps.\\
        
        \textbf{Else Branch}
        \item Once $m > n$, the else branch will execute. This occurs after n iterations, thus the else branch is iterated through $(m - n)$ times.
        \item For each iteration, \texttt{lst.\_\_contains\_\_(0)} is called. After the first if branch iterations, the entire list is swapped back and forth until it is ordered in the same way it started.
        \item Calling \texttt{lst.\_\_contains\_\_(0)} will call the first node in the list. This executes the elif branch in the \texttt{\_\_contains\_\_} method, which takes constant time. This gves a total of $(m - n)$ steps.\\
        
        \textbf{Total Running Time}
        \item Let $T_2$ represent the total running time of Heuristic 2
        \item Ignoring constant factors, the total running time is $T_2 = 2n - \frac{n(n+2)}{2} + n^2 + (m - n)$.\\
    \end{itemize}
    
    \item[c)] \textbf{Comparing $T_1$ and $T_2$}
    \begin{itemize}
        \item We will compare the running times $T_1$ and $T_2$ to determine whether $T_1 - T_2 \notin \mathcal{O}(1)$.  
        \begin{align}
            & T_1 - T_2\\
            &= (n^2 + (m - n)) - (2n - \frac{n(n+2)}{2} + n^2 + (m - n))\\
            &= n^2 - 2n + \frac{n(n+2)}{2} - n^2\\
            &= \frac{n(n+2)}{2} - 2n\\
            &= \frac{n^2 + 2n - 4n}{2}\\
            &= \frac{n^2 - 2n}{2}
        \end{align}
        \item Thus, $T_1 - T_2 \in \mathcal{O}(n^2)$, and $\notin \mathcal{O}(1)$. 
    \end{itemize}
\end{enumerate}
\end{enumerate}

\section*{Part 2: Linked List Visualization}
Complete this part in the provided \texttt{a1\_part2.py} starter file.
Do \textbf{not} include your solution in this file.

\end{document}
